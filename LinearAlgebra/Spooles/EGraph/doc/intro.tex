\chapter{{\tt EGraph}: Element Graph Object}
\par
The {\tt EGraph} object is used to model a graph that has a natural
element structure (as from finite elements) or a natural covering
clique structure (e.g., the rows of $A$ are natural cliques for
the graph of $A^TA$).
\par
Translating an element graph {\tt EGraph} object 
into an adjacency list {\tt Graph} object is an easy task ---
we provide a method to do so --- but the process in reverse is much
more difficult.
Given a {\tt Graph} object, it is simple to construct a trivial 
element graph object, simply take each $(i,j)$ edge to be an element.
Constructing an element graph with a smaller number of elements 
is more difficult.
\par
Element graphs, when they arise naturally or are constructed from
an adjacency graph, have great potential. 
The element model for sparse elimination {\it appears} to be more
powerful than the vertex adjacency list model in the sense that
concepts like indistinguishability, outmatching and deficiency are
more naturally defined with elements.
An element graph might be a more natural vehicle for partitioning
graphs, because if one consider elements as the ``nodes'' in a
Kernighan-Lin type algorithm, then the ``edge'' separators are
formed of vertices of the original graph.
