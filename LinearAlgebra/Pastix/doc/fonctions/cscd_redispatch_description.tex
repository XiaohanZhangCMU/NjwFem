\descript{
           Redistribute the first cscd, distributed with \texttt{l2g}
           local to global array, into a new one using \texttt{dl2g}
           as local to global array.
}
         {
           Redistribute the first cscd, distributed with \texttt{l2g}
           local to global array, into a new one using \texttt{dl2g}
           as local to global array.
           
           The algorithm works in four main steps :
           \begin{itemize}
           \item gather all new loc2globs on all processors;
           \item allocate \texttt{dia}, \texttt{dja} and \texttt{da};
           \item Create new CSC for each processor and send it;
           \item Merge all new CSC to the new local CSC with \texttt{cscd\_addlocal()}.
           \end{itemize}
           
           If communicator size is one, check that $n = dn$ and 
           $l2g = dl2g$ and simply create a copy of the first CSCD.

           In Fortran the function as to be called in to step, the
           first one, \texttt{CSCD\_REDISPATCH\_FORTRAN}, to compute the
           new CSCD, and the second one,
           \texttt{CSCD\_REDISPATCH\_FORTRAN\_END} to copy the
           computed CSCD into the user allocated structure.
}
