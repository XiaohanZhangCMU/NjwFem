%\documentclass[landscape,letterpaper]{article}
\documentclass{article}
\usepackage{multicol}
\usepackage{listings}
\usepackage{parskip}
\usepackage[landscape]{geometry}
\usepackage[usenames,dvipsnames]{color} 
\usepackage{url}
\usepackage{amsmath}
\usepackage{graphics}
\usepackage{verbatim} 
%\usepackage{a4}%-Northamerica: {fullpage}
\usepackage{svn}%!/hg/s/linux/9/app/teTeX/11.05/texmf-dist/tex/latex/svn/svn.sty
%           --- ubuntu lacks this but has latex-svinfo - usepackage{svninfo}
\addtolength{\textheight}{52mm}
\addtolength{\topmargin}{-21mm}
\addtolength{\oddsidemargin}{-37mm}
\addtolength{\evensidemargin}{-37mm}
\addtolength{\textwidth}{74mm}
%\setlength{\columnsep}{5pt}		% default=10pt
\setlength{\columnseprule}{0pt}		% default=0pt (no line)

\newcommand\pastix{\textsc{PaStiX}}

%%\newcommand{\type}[1]{\textcolor{OliveGreen}{#1}}
%%\newcommand{\var}[1]{\textcolor{YellowOrange}{#1}}
%%\newcommand{\func}[1]{\textcolor{Blue}{#1}}
%%\newcommand{\st}{\raisebox{-0.5ex}{*}}

\newcommand{\type}[1]{\textbf{#1}}
\newcommand{\var}[1]{\texttt{#1}}
\newcommand{\func}[1]{\texttt{#1}}
\newcommand{\forkey}[1]{\texttt{#1}}
\newcommand{\st}{\raisebox{-0.5ex}{*}}


\newenvironment{tabI}{\begin{tabular}{p{30mm}p{96mm}}}{\end{tabular}}
\newenvironment{tabIbis}[1]{\begin{tabular}{|p{33mm}|p{3mm}|p{73mm}|}\hline\multicolumn{3}{|l|}{#1}\\\hline}{\end{tabular}}
\newenvironment{tabIter}{\begin{tabular}{|p{45mm}|p{15mm}|p{155mm}|p{40mm}|p{15mm}|}}{\end{tabular}}
\newenvironment{tabII}{\begin{tabular}{|p{86mm}|p{8mm}|p{10mm}|p{12mm}|}}{\end{tabular}}
\newenvironment{tabIII}{\begin{tabular}{|p{32mm}|p{90mm}|}}{\end{tabular}}
\newenvironment{tabIIIbis}{\begin{tabular}{|p{35mm}|p{85mm}|}}{\end{tabular}}
\newenvironment{tabIV}{\begin{tabular}{|p{60mm}|p{60mm}|}}{\end{tabular}}
\newenvironment{tabTit}[1]{\underline{\textbf{#1}}\\ \begin{tabI}}{\end{tabI}}
\newenvironment{mylisting}
{\begin{list}{}{\setlength{\leftmargin}{1em}}\item\scriptsize\bfseries}
{\end{list}}

\newenvironment{mytinylisting}
{\begin{list}{}{\setlength{\leftmargin}{1em}}\item\tiny\bfseries\begin{verbatim}}
{\end{list}\end{verbatim}}


\lstnewenvironment{functionc}{
\lstset{language=C}
\lstset{emph={pastix_data_t,pastix_float_t,pastix_int_t,MPI_Comm,PASTIX_PTR,PASTIX_INT,PASTIX_FLOAT,CSCD_OPERATIONS_t},emphstyle=\bfseries}
}{}

\newenvironment{arguments}{\begin{tabI}}{\end{tabI}}
\newenvironment{arguments2}{\begin{tabular}{p{10mm}p{30mm}p{86mm}}}{\end{tabular}}
\newenvironment{returns}{Returns~:\\\begin{tabI}}{\end{tabI}}

\newcommand{\descript}[2]{\par#1}

\newcommand*{\tabIIItitle}[1]{\multicolumn{2}{|l|}{#1}}
%
\newcommand{\Sect}[1]{\par\noindent\medskip\normalsize\textbf{#1}
  \vskip -.2ex plus 1ex minus 1ex \small}
\newcommand{\Sectbis}[2]{\par\noindent\medskip\large\textbf{#1}\small\ #2
  \vskip -.2ex plus 1ex minus 1ex \small}

\newcommand{\Sectii}[1]{\par\noindent\medskip\normalsize\textbf{#1}
  \vskip -.2ex plus 1ex minus 1ex \small}

\newcommand{\Sectiii}[1]{\par\noindent\medskip\small\textbf{#1}
  \hrule\small}

\newcommand*{\Ecmd}[1]{$\left\langle \textrm{#1} \right\rangle$}
\newcommand*{\sEcmd}[1]{{\small\Ecmd{#1}}}
\newcommand*{\RET}[0]{\Ecmd{\textsc{ret}}}
\newcommand*{\TAB}[0]{\Ecmd{\textsc{tab}}}

\raggedcolumns%\raggedbottom
\setlength{\columnseprule}{.4pt}% default 0
\setlength{\columnsep}{22pt}% default is less (18 pt?)
\pagestyle{empty}


\begin{document}

%\svnid{$Id$}

\SVN $Date$
%% \svnidlong
%% {$HeadURL$}
%% {$LastChangedDate$}
%% {$LastChangedRevision$}
%% {$LastChangedBy$}

%\begin{multicols}{1}
%%%%%%%%%%%%%%%%%%%%%%%%%%%%%%%%%%%%
%%%%%%% CLASSICAL PASTIX   %%%%%%%%%
%%%%%%%%%%%%%%%%%%%%%%%%%%%%%%%%%%%%

\begin{center}
  {\LARGE \pastix{} version 5.1 Quick Reference Guide}
  %  {\small \pastix{} Version 5.1}% {\footnotesize --- needs \em{more} updating!}}
  \\[1ex] {\tiny \SVNDate}%\SVNRev}
  %           \footnotesize \today
\end{center}

%\begin{enumerate}
%\item  \textsc{Nota Bene:} S is the \emph{language},
%  R is one \emph{dialect}!
%\item This is a list of the more widely used \textbf{key - shortcuts}.
%  Many more are available, and most are accessible from the Emacs
%  \textbf{Menus} such as \texttt{iESS}, \texttt{ESS}, etc.
%NN  \vspace*{-3ex}
%\end{enumerate}
%\end{multicols}
%NN \rule{\textwidth}{.2pt}%----------------------------------------------------
\begin{multicols}{2}
  \Sect{\pastix{} Calls with global matrix }
  
  \hrule
  \lstset{language=C}
  \lstset{emph={pastix_data_t,pastix_float_t,pastix_int_t,MPI_Comm,PASTIX_PTR,PASTIX_INT,PASTIX_FLOAT},emphstyle=\bfseries}
%%  \begin{lstlisting}
%%#include "pastix.h"
%%
%%void pastix(pastix_data_t **pastix_data, MPI_Comm pastix_comm, 
%%            pastix_int_t    n,           pastix_int_t   *colptr, 
%%            pastix_int_t   *row,         pastix_float_t *avals, 
%%            pastix_int_t   *perm,        pastix_int_t   *invp,  
%%            pastix_float_t *b,           pastix_int_t    rhs,
%%            pastix_int_t   *iparm,       double         *dparm);
%%  \end{lstlisting}
%%  \textcolor{Rhodamine}{\#include} \textcolor{Peach}{``pastix.h''}\\
\texttt{\#include ``pastix.h''}

\preproc{\#include} \inc{``pastix.h''}

\begin{type-func-type-mod-var-type-mod-var}
  \tftmv{void}{pastix}{pastix\_data\_t }{\st{}\st{}}{pastix\_data},
                  \tmv{MPI\_Comm       }{          }{pastix\_comm},\\
                  \tmv{pastix\_int\_t  }{          }{n},
                  \tmv{pastix\_int\_t  }{\st{}     }{colptr},\\
                  \tmv{pastix\_int\_t  }{\st{}     }{row},
                  \tmv{pastix\_float\_t}{\st{}     }{avals},\\ 
                  \tmv{pastix\_int\_t  }{\st{}     }{perm},
                  \tmv{pastix\_int\_t  }{\st{}     }{invp},\\
                  \tmv{pastix\_float\_t}{\st{}     }{b},
                  \tmv{pastix\_int\_t  }{          }{rhs},\\
                  \tmv{pastix\_int\_t  }{\st{}     }{iparm},
                  \tmv{double          }{\st{}     }{dparm} );\\
  \label{proto:pastix}
\index{Fonctions@Fonctions!\texttt{pastix}}
\end{type-func-type-mod-var-type-mod-var}

\begin{tabular}{l@{ }l@{ }l@{ }l@{ }l@{ }l@{ }l@{ }l@{ }l@{ }}
  \type{void} & \func{pastix} ( & \type{pastix\_data\_t}  & \st{}\st{} & \var{pastix\_data},&
                                  \type{MPI\_Comm}        &            & \var{pastix\_comm},\\
              &                 & \type{pastix\_int\_t}   &            & \var{n},&
                                  \type{pastix\_int\_t}   & \st{}      & \var{colptr},\\
              &                 & \type{pastix\_int\_t}   & \st{}      & \var{row},&
                                  \type{pastix\_float\_t} & \st{}      & \var{avals},\\ 
              &                 & \type{pastix\_int\_t}   & \st{}      & \var{perm},&
                                  \type{pastix\_int\_t}   & \st{}      & \var{invp},\\
              &                 & \type{pastix\_float\_t} & \st{}      & \var{b},&
                                  \type{pastix\_int\_t}   &            & \var{rhs},\\
              &                 & \type{pastix\_int\_t}   & \st{}      & \var{iparm},&
                                  \type{double}           & \st{}      & \var{dparm} );\\
  \label{proto:pastix}
\end{tabular}

  \vskip 0.3 cm
  \hrule

\texttt{\#include ``pastix\_fortran.h''}

\begin{tabular}{ll}
  \type{pastix\_data\_ptr\_t} &  \var{pastix\_data}\\
  \type{integer}              &  \var{pastix\_comm}\\
  \type{pastix\_int\_t}       &  \var{n}, \var{rhs}, \var{ia(n)}, \var{ja(nnz)}\\
  \type{pastix\_float\_t}     &  \var{avals(nnz)}, \var{b(n)}\\
  \type{pastix\_int\_t}       &  \var{perm(n)}, \var{invp(n)}, \var{iparm(64)}\\
  \type{real}\st{}8           &  \var{dparm(64)}\\
\end{tabular}

\begin{tabular}{l@{ }l}
  \forkey{call} \func{pastix\_fortran} ( & \var{pastix\_data},
  \var{pastix\_comm}, \var{n}, \var{ia}, \var{ja}, \var{avals},\\
                       & \var{perm}, \var{invp}, \var{b}, \var{rhs},
  \var{iparm}, \var{dparm} )\\
\end{tabular}
%% 
%%   \lstset{emph={pastix_data_ptr_t,pastix_int_t,pastix_float_t},emphstyle=\bfseries}
%%   \lstset{language=Fortran}
%%   \begin{lstlisting}
%% #include "pastix_fortran.h"
%%   pastix_data_ptr_t  pastix_data
%%   integer            pastix_comm
%%   pastix_int_t       n, rhs, ia(n), ja(nnz)
%%   pastix_float_t     avals(nnz), b(n)
%%   pastix_int_t       perm(n), invp(n), iparm(64)
%%   real*8             dparm(64)    
%%     ...
%% call pastix_fortran(pastix_data, pastix_comm, n, ia, ja, avals,
%%                     perm, invp, b, rhs, iparm, dparm)
%%   \end{lstlisting}
%% 
  \hrule
  %\begin{multicols}{2}
  \begin{tabI}
    \texttt{pastix\_data} & Data structure used to keep informations for a
    step by step call. Should be given unallocated for first call.\\
    %Next calls should use the same structure.\\
    \texttt{pastix\_comm} & MPI communicator used to solve the system.\\
    \texttt{n}            & Matrix dimension.\\
    \texttt{nnz}          & Number of non-zeros.\\
    \texttt{colptr}, \texttt{row},       \texttt{avals} 
    & Matrix in CSC format (see example below).\\
    
    \texttt{perm}         & Permutation vector. \\
    \texttt{invp}         & Inverse permutation vector.\\
    \texttt{b}            & Right-hand side(s) and solution(s) as output.\\
    \texttt{rhs}          & Number of right-hand side(s).\\
    \texttt{iparm}        & Integer parameter vector.\\
    \texttt{dparm}        & Double parameter vector.\\
  \end{tabI}\\
  
  \hrule
  In the current release, the matrix must be given in a Compress Sparse Column format in fortran numbering (starts from 1).
  
  \begin{multicols}{2}
    CSC matrix example :
    $  
    \begin{pmatrix} 
      1 & 0 & 0 & 0 & 0\\
      0 & 3 & 0 & 0 & 0\\ 
      2 & 0 & 5 & 0 & 0\\  
      0 & 4 & 6 & 7 & 0\\
      0 & 0 & 0 & 0 & 8\\
    \end{pmatrix}
    $\\
    \begin{tabular}{lcl}
      colptr & = & $\{1,3,5,7,8,9\}$\\
      row    & = & $\{1,3,2,4,3,4,4,5\}$\\
      avals  & = & $\{1,2,3,4,5,6,7,8\}$\\
    \end{tabular}
  \end{multicols}

  \vskip 3 cm

%%  \begin{center}
%%    {\LARGE \pastix{} with distributed matrix}
%%  \end{center}

  %\Sect{Work in progress...}

%%%%%%%%%%%%%%%%%%%%%%%%%%%%%%%%%%%%
%%%%%%% DISTRIBUTED MATRIX %%%%%%%%%
%%%%%%%%%%%%%%%%%%%%%%%%%%%%%%%%%%%%

  \Sect{\pastix{} Calls with distributed matrix}

  \hrule
  \lstset{language=C}
  \lstset{emph={pastix_data_t,pastix_float_t,pastix_int_t,MPI_Comm},emphstyle=\bfseries}

  \begin{lstlisting}
#include "pastix.h"

void dpastix(pastix_data_t **pastix_data, MPI_Comm       pastix_comm, 
             pastix_int_t    n,           pastix_int_t   *colptr, 
             pastix_int_t   *row,         pastix_float_t *avals, 
             pastix_int_t   *loc2glb,    
             pastix_int_t   *perm,        pastix_int_t   *invp,        
             pastix_float_t *b,           pastix_int_t    rhs,         
             pastix_int_t   *iparm,       double         *dparm);
  \end{lstlisting}

  \hrule
  \lstset{emph={pastix_dtat_ptr_t,pastix_int_t,pastix_float_t},emphstyle=\bfseries}
  \lstset{language=Fortran}
  \begin{lstlisting}
#include "pastix_fortran.h"
  pastix_data_ptr_t  pastix_data
  integer            mpi_comm
  pastix_int_t       n, rhs, ia(n), ja(nnz)
  pastix_float_t     avals(nnz), b(n)
  pastix_int_t       loc2glb(n), perm(n), invp(n), iparm(64)
  real*8             dparm(64)    
    ...
call dpastix_fortran(pastix_data, mpi_comm, n, ia, ja, avals,
                     loc2glb, perm, invp, b, rhs, iparm, dparm) 
  \end{lstlisting}
  \hrule
  
  Additional parameter :

  \begin{tabI}
    \texttt{loc2glb} & Local to global column number correspondance,\\
                     & all columns must be distributed once and loc2glob must be ordered increasingly.\\
  \end{tabI}\\
  \hrule
  %% The only parameters changing from classical \pastix{} call are
  %% the ones corresponding to the CSC matrix.\\
  %% The CSC matrix is now distributed.\\

  %% \begin{tabI}
  %%   \texttt{colptr}  & Tabular indicating start of each column in row and avals tabulars.\\
  %%   \texttt{row}     & Line indexes for each local column.\\
  %%   \texttt{avals}   & Values of the local part of the matrix.\\
  %%   \texttt{loc2glb} & global number of each local columns.
  %% \end{tabI}
  The distribution of the CSC matrix is given through the loc2glb vector (see example below).

  \begin{multicols}{2}
    dCSC matrix example :\\

    $  
    \begin{pmatrix} 
      P_1 & P_2 & P_1 & P_2 & P_1\\
      1   & 0   & 0   & 0   & 0\\
      0   & 3   & 0   & 0   &  0\\ 
      2   & 0   & 5   & 0   & 0\\  
      0   & 4   & 6   & 7   & 0\\
      0   & 0   & 0   & 0   & 8\\
    \end{pmatrix}
    $\\
    \vskip 2 cm
    On processor one : \\
    \begin{tabular}{lcl}
      colptr  & = & $\{1,3,5,6\}$\\
      row     & = & $\{1,3,3,4,5\}$\\
      avals   & = & $\{1,2,5,6,8\}$\\
      loc2glb & = & $\{1,3,5\}$\\
    \end{tabular}

    On processor two : \\
    \begin{tabular}{lcl}
      colptr   & = & $\{1,3,4\}$\\
      row      & = & $\{2,4,4\}$\\
      avals    & = & $\{3,4,7\}$\\
      loc2glb  & = & $\{2,4\}$\\
    \end{tabular}
  \end{multicols}
\end{multicols}
\newpage

\Sect{Floating and Integer parameters (dparm and iparm)}
%%    ~~~~~~~~~~~~~~~~~~~~~~~~~~~~~~
\input{dparm.tex}
%%    ~~~~~~~~~~~~~~~~~~~~~~~~~~~~~~
\input{iparm.tex}
\newpage

\Sect{\pastix{} API : Macros}
%%    ~~~~~~~~~~~~~~~~~~~~~~~~~~~~~~
\begin{multicols}{2}

\input{api.tex}
\end{multicols}
\newpage
\Sect{\pastix{} API : Functions}
\begin{multicols}{2}
\input{apifunctions.tex}
\end{multicols}
\newpage
\Sect{\pastix{} API : Murge Interface}
\input{murge.tex}

\newpage
  %%%%%%%%%%%%%%%%%%%%%%%%%%%%%%%%%%%%
  %%%%%%% COMPILATION        %%%%%%%%%
  %%%%%%%%%%%%%%%%%%%%%%%%%%%%%%%%%%%%

\begin{multicols}{2}

  \begin{center}
    {\LARGE How-to compile \pastix{}}
  \end{center}
  \Sect{Requirements}
  The \pastix{} team recommands you to get Scotch
  (\url{http://gforge.inria.fr/projects/scotch/}) and compile it.\\ 
  Then go into \pastix{} directory. Select the config file corresponding
  to your machine in \texttt{\$\{PASTIX\_DIR\}/config/} and copy it to
  \texttt{\$\{PASTIX\_DIR\}/config.in}.\\ 
  Now edit this file, select the options you want, and set the correct
  path for \texttt{\$\{SCOTCH\_HOME\}}.\\ 
  If you want to use METIS, you also have to compile it and edit the
  path in \texttt{config.in}.

  \Sect{Compilation}

  \begin{tabIII}
    \hline
    \tabIIItitle{Makefile tags (from the root directory)}\\
    \hline
    \texttt{make clean}     & clean to rebuild the library\\
    \texttt{make expor}     & compile the \pastix{} library\\
    \texttt{make debug}     & compile the \pastix{} library in debug mode\\
    \texttt{make install}   & install the \pastix{} library 
    (\texttt{'make expor'} or \texttt{'debug'} required)\\
    \texttt{make drivers}      & compile the matrix drivers
    (\texttt{'make install'} required)\\
    \texttt{make examples}  & compile the PaStiX examples
    (\texttt{'make install'} required)\\
    \hline 
    \hline 
    \texttt{make murge\_examples} & compile MURGE examples
    (only available in distributed mode -DDISTRIBUTED, \texttt{'make install'} required)\\
    \hline
  \end{tabIII}

  \Sect{Compilation options (\texttt{config.in})}

  \begin{tabIII}
    \hline
    \tabIIItitle{General options} \\
    \hline
    \texttt{-DDISTRIBUTED}         & Enable distributed mode \texttt{dpastix} (PT-Scotch required)\\
    \texttt{-DFORCE\_LONG}         & Use long integers\\
    \texttt{-DFORCE\_DOUBLE}       & Use double floating coefficients\\
    \texttt{-DFORCE\_COMPLEX}      & Use complex coefficients\\
    \texttt{-DFORCE\_NOMPI}        & Compilation without MPI support\\
    \texttt{-DFORCE\_NOSMP}        & Compilation without Thread support\\
    %  \texttt{-DONLY\_LOAD\_SYMBMTX} & Will only load the symbol matrix and make a fake factorization\\
    \hline
  \end{tabIII}

  \begin{tabIII}
    \hline
    \tabIIItitle{Preprocessing options} \\
    \hline
    \texttt{-DMETIS}           & Use Metis ordering library (needs \texttt{-L\$\{METIS\_HOME\} -lmetis)}\\
    \texttt{-DWITHOUT\_SCOTCH} & Deactivate Scotch ordering library\\
    \hline
  \end{tabIII}

  \begin{tabIII}
    \hline
    \tabIIItitle{Solver options - \textit{See \texttt{\$PASTIX\_HOME/sopalin/src/sopalin\_define.h}}}\\
    \hline
    \texttt{-DNUMA\_ALLOC}      & Allocation of the coefficient vector locally on each thread.\\
    \texttt{-DNO\_MPI\_TYPE}    & Avoids MPI types usage by copying into communication buffers.\\
    \texttt{-DTEST\_IRECV}      & Non blocking receptions\\
    \texttt{-DTHREAD\_COMM}     & Reception on dedicated threads (persistent communications).\\
    \texttt{-DPASTIX\_FUNNELED} & Main thread makes all communications.\\
    %  \texttt{-DNOSMP\_RAFF}     & Uses threads in refinement functions.\\
    \hline 
  \end{tabIII}

  \begin{tabIII}
    \hline
    \tabIIItitle{Statistics and Debug options - \textit{See \texttt{\$PASTIX\_HOME/sopalin/src/sopalin\_define.h}}}\\
    \hline
    \texttt{-DMEMORY\_USAGE}   & Shows memory allocations (could slow down execution)\\
    \texttt{-DSTATS\_SOPALIN}  & Shows parallelization memory overhead\\
    \texttt{-DVERIF\_MPI}      & Checks MPI Communication successful\\
    \texttt{-DFLAG\_ASSERT}    & Adds some checks during factorization\\
%    \texttt{-DTRACE\_SOPALIN}  & Generate a trace file for PAJE\\
    \hline
  \end{tabIII}
\vskip 6 cm

%%%%%%%%%%%%%%%%%%%%%%%%%%%%%%%%%%%%
%%%%%%% Details            %%%%%%%%%
%%%%%%%%%%%%%%%%%%%%%%%%%%%%%%%%%%%%

  \input{details.tex}


  %%%%%%%%%%%%%%%%%%%%%%%%%%%%%%%%%%%%
  %%%%%%% ADVANCED           %%%%%%%%%
  %%%%%%%%%%%%%%%%%%%%%%%%%%%%%%%%%%%%
  \newpage
  \input{advanced.tex}
\end{multicols}
\end{document}
